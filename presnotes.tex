\documentclass{article}

\begin{document}
\section{?`Preguntas\ldots?}
\label{sec:questions}

\begin{enumerate}
\item What kind of telescope is Apache?
  \begin{enumerate}
  \item Not sure if optical or not,
  but I do know that it was specifically crafted to be able to image
  large swaths of the sky: due to it's double corrective lens design,
  the telescope can take sharply focused images from an area of three
  degrees, equal to the diameter of about 30 full moons.
  \end{enumerate}
\item Intro to Sloan? Transition between Sloan and galaxy spectra?
  Any immediate relevance?
  \begin{enumerate}
  \item From where do we get these galaxy spectra?  From SDSS, of
    course.
  \item Using the vast amount of galaxy data from SDSS release 3,
    astronomers have come up with a set of 10 galaxy ``spectra'' that
    are useful for categorizing real galaxy spectra.  Rather than handling an entire spectrum, which details the
    flux as a function of wavelength, we can describe a spectrum by a
    set of 10 numbers which are its coefficients with respect to this
    set of 10.
  \item Analogous to\ldots
  \end{enumerate}
\item Number of supernovae found by Or's slow process: 104
\end{enumerate}

\section{Suggestions for Final Slides}
\label{sec:changes}

\begin{enumerate}
\item Put the date on the title page
\item What should we include in our progress report?
  \begin{enumerate}
  \item Learn how to use a computer!!! (i.e., how to use the terminal in
    Linux)
  \item Python and a few modules (numpy, matplotlib)
  \item Linear algebra to understand PCA
  \item Developing our own modules that automate the PCA process
  \end{enumerate}
\item Looking ahead\ldots
  \begin{enumerate}
  \item Once we finish debugging our code, we will begin to seriously
    analyze our results
  \item More user-friendly code: learn about object-oriented
    programming
  \end{enumerate}

\end{enumerate}
\end{document}
